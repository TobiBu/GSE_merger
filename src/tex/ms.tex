%% This is emulateapj reformatting of the AASTEX sample document
%%
%\documentclass[apj,twocolappendix,numberedappendix,appendixfloats]{emulateapj}
\documentclass[useAMS,usenatbib]{mnras}
\usepackage{showyourwork}

\usepackage{natbib}
%\bibliographystyle{apj}
\bibliographystyle{mnras}

\usepackage{tabu,booktabs}
\usepackage{amsmath}
\usepackage{xcolor}
\usepackage[english]{babel}
\usepackage{graphicx}
\usepackage{xspace}

\usepackage{longtable}
\usepackage{threeparttablex}
\usepackage{array}

% --- HACK to remove showyourwork abstract margin icon
\makeatletter
\def\abstract{\if@twocolumn
   \start@SFBbox\@abstract
 \else
   \@abstract
 \fi}
\def\endabstract{\if@twocolumn
   \endlist\finish@SFBbox
 \else
  \endlist
 \fi}
\makeatother
% ---

%\usepackage{pdflscape}

%%%%% AUTHORS - PLACE YOUR OWN MACROS HERE %%%%%

\def \etal {et~al.~}
\newcommand{\MKN}[1]{{\color{blue}{\bf MKN}:~ #1}}

\newcommand{\Rmax}{$R_{\rm max}$}
\newcommand{\Vmax}{$V_{\rm max}$}
\newcommand{\nPS}{$n_{\rm P\&S}$}
\newcommand{\nEin}{$n_{\rm E}$}
\newcommand{\amiga}{\texttt{AMIGA}}
\newcommand{\ahf}{\texttt{AHF}}
\newcommand{\mlapm}{\texttt{MLAPM}}
\newcommand{\hMpc}{{\ifmmode{h^{-1}{\rm Mpc}}\else{$h^{-1}$Mpc}\fi}}
\newcommand{\Mpc}{{\ifmmode{{\rm Mpc}}\else{Mpc}\fi}}
\newcommand{\hkpc}{{\ifmmode{h^{-1}{\rm kpc}}\else{$h^{-1}$kpc}\fi}}
\newcommand{\kpc}{{\ifmmode{ {\rm kpc} }\else{{\rm kpc}}\fi}}
\newcommand{\kms}{{\ifmmode{ {\rm km\,s^{-1}} }\else{ ${\rm km\,s^{-1}}$ }\fi}}
\newcommand{\hMsun}{{\ifmmode{h^{-1}{\rm {M_{\odot}}}}\else{$h^{-1}{\rm{M_{\odot}}}$}\fi}}
\newcommand{\Msun}{{\ifmmode{{\rm M}_{\odot}}\else{${\rm M}_{\odot}$}\fi}}
\newcommand{\Mhalo}{{\ifmmode{M_{\rm halo}}\else{$M_{\rm halo}$}\fi}}
\newcommand{\Rvir}{{\ifmmode{R_{\rm vir}}\else{$R_{\rm vir}$}\fi}}
\newcommand{\Mvir}{{\ifmmode{M_{\rm vir}}\else{$M_{\rm vir}$}\fi}}
\newcommand{\Mstar}{{\ifmmode{M_{\rm star}}\else{$M_{\rm star}$}\fi}}
\newcommand{\Vrot}{{\ifmmode{V_{\rm rot}}\else{$V_{\rm rot}$}\fi}}
\newcommand{\ltsima}{$\; \buildrel < \over \sim \;$}
\newcommand{\gtsima}{$\; \buildrel > \over \sim \;$}
\newcommand{\lsim}{\lower.5ex\hbox{\ltsima}}
\newcommand{\gsim}{\lower.5ex\hbox{\gtsima}}
\def\nbody{$N$-body}
\def\lesssim{\mathrel{\hbox{\rlap{\hbox{\lower4pt\hbox{$\sim$}}}\hbox{$<$}}}}
\def\gtrsim{\mathrel{\hbox{\rlap{\hbox{\lower4pt\hbox{$\sim$}}}\hbox{$>$}}}}
\newcommand{\new}[1]{\textbf{\textcolor{blue}{#1}}}
\newcommand{\problem}[1]{\textbf{\textcolor{red}{#1}}}
\newcommand{\Sec}[1]{Section~\ref{#1}}
\newcommand{\Eq}[1]{Eq.~(\ref{#1})}
\newcommand{\Fig}[1]{Fig.~\ref{#1}}
\newcommand{\beq}{\begin{equation}}
\newcommand{\eeq}{\end{equation}}
\def\beqa{\begin{eqnarray}}
\def\eeqa{\end{eqnarray}}
\def\LCDM{\ensuremath{\Lambda}CDM}
\def\LWDM{\ensuremath{\Lambda}WDM}
\def\head{ \vbox to 0pt{\vss \hbox to 0pt{\hskip 440pt\rm
      LA-UR-10-07069\hss} \vskip 25pt}}

%%UNITS
\def \kms {\ifmmode  \,\rm km\,s^{-1} \else $\,\rm km\,s^{-1}  $ \fi }
\def \kpc {\ifmmode  {\rm kpc}  \else ${\rm  kpc}$ \fi  }  
\def \hkpc {\ifmmode  {h^{-1}\rm kpc}  \else ${h^{-1}\rm kpc}$ \fi  }  
\def \hMpc {\ifmmode  {h^{-1}\rm Mpc}  \else ${h^{-1}\rm Mpc}$ \fi  }  
\def \Mpch {\ifmmode  {h^{-1}\rm Mpc}  \else ${h^{-1}\rm Mpc}$ \fi  }  
\def \Msun {\ifmmode {\rm M}_{\odot} \else ${\rm M}_{\odot}$ \fi} 
\def \hMsun {\ifmmode h^{-1}\,\rm M_{\odot} \else $h^{-1}\,\rm M_{\odot}$ \fi}

%%COSMOLOGY
\def \LCDM {\ifmmode \Lambda{\rm CDM} \else $\Lambda{\rm CDM}$ \fi}
\def \sig8 {\ifmmode \sigma_8 \else $\sigma_8$ \fi} 
\def \OmegaM {\ifmmode \Omega_{\rm m} \else $\Omega_{\rm m}$ \fi} 
\def \Omegab {\ifmmode \Omega_{\rm b} \else $\Omega_{\rm b}$ \fi} 
\def \OmegaL {\ifmmode \Omega_{\rm \Lambda} \else $\Omega_{\rm \Lambda}$\fi} 
\def \Deltavir {\ifmmode \Delta_{\rm vir} \else $\Delta_{\rm vir}$ \fi}
\def \rhocrit {\ifmmode \rho_{\rm crit} \else $\rho_{\rm crit}$ \fi}
\def \rhou {\ifmmode \rho_{\rm u} \else $\rho_{\rm u}$ \fi}
\def \zc {\ifmmode z_{\rm c} \else $z_{\rm c}$ \fi}

\def\lcdm{\ensuremath{\Lambda\textrm{CDM}}\xspace}  
\def\omegam{\ensuremath{\Omega_\textrm{m}}\xspace}
\def\omegal{\ensuremath{\Omega_\Lambda}\xspace}
\def\omegab{\ensuremath{\Omega_\textrm{b}}\xspace}
\def\omegar{\ensuremath{\Omega_\textrm{r}}\xspace}
%%%%% AUTHORS - PLACE YOUR OWN MACROS HERE %%%%%

%\usepackage{tabu,booktabs}

\def\aj{AJ}%
         % Astronomical Journal
\def\actaa{Acta Astron.}%
         % Acta Astronomica
\def\araa{ARA\&A}%
         % Annual Review of Astron and Astrophys
\def\apj{ApJ}%
         % Astrophysical Journal
\def\apjl{ApJ}%
         % Astrophysical Journal, Letters
\def\apjs{ApJS}%
         % Astrophysical Journal, Supplement
\def\ao{Appl.~Opt.}%
         % Applied Optics
\def\apss{Ap\&SS}%
         % Astrophysics and Space Science
\def\aap{A\&A}%
         % Astronomy and Astrophysics
\def\aapr{A\&A~Rev.}%
         % Astronomy and Astrophysics Reviews
\def\aaps{A\&AS}%
         % Astronomy and Astrophysics, Supplement
\def\azh{AZh}%
         % Astronomicheskii Zhurnal
\def\baas{BAAS}%
         % Bulletin of the AAS
\def\bac{Bull. astr. Inst. Czechosl.}%
         % Bulletin of the Astronomical Institutes of Czechoslovakia
\def\caa{Chinese Astron. Astrophys.}%
         % Chinese Astronomy and Astrophysics
\def\cjaa{Chinese J. Astron. Astrophys.}%
         % Chinese Journal of Astronomy and Astrophysics
\def\icarus{Icarus}%
         % Icarus
\def\jcap{J. Cosmology Astropart. Phys.}%
         % Journal of Cosmology and Astroparticle Physics
\def\jrasc{JRASC}%
         % Journal of the RAS of Canada
\def\mnras{MNRAS}%
         % Monthly Notices of the RAS
\def\memras{MmRAS}%
         % Memoirs of the RAS
\def\na{New A}%
         % New Astronomy
\def\nar{New A Rev.}%
         % New Astronomy Review
\def\pasa{PASA}%
         % Publications of the Astron. Soc. of Australia
\def\pra{Phys.~Rev.~A}%
         % Physical Review A: General Physics
\def\prb{Phys.~Rev.~B}%
         % Physical Review B: Solid State
\def\prc{Phys.~Rev.~C}%
         % Physical Review C
\def\prd{Phys.~Rev.~D}%
         % Physical Review D
\def\pre{Phys.~Rev.~E}%
         % Physical Review E
\def\prl{Phys.~Rev.~Lett.}%
         % Physical Review Letters
\def\pasp{PASP}%
         % Publications of the ASP
\def\pasj{PASJ}%
         % Publications of the ASJ
\def\qjras{QJRAS}%
         % Quarterly Journal of the RAS
\def\rmxaa{Rev. Mexicana Astron. Astrofis.}%
         % Revista Mexicana de Astronomia y Astrofisica
\def\skytel{S\&T}%
         % Sky and Telescope
\def\solphys{Sol.~Phys.}%
         % Solar Physics
\def\sovast{Soviet~Ast.}%
         % Soviet Astronomy
\def\ssr{Space~Sci.~Rev.}%
         % Space Science Reviews
\def\zap{ZAp}%
         % Zeitschrift fuer Astrophysik
\def\nat{Nature}%
         % Nature
\def\iaucirc{IAU~Circ.}%
         % IAU Cirulars
\def\aplett{Astrophys.~Lett.}%
         % Astrophysics Letters
\def\apspr{Astrophys.~Space~Phys.~Res.}%
         % Astrophysics Space Physics Research
\def\bain{Bull.~Astron.~Inst.~Netherlands}%
         % Bulletin Astronomical Institute of the Netherlands
\def\fcp{Fund.~Cosmic~Phys.}%
         % Fundamental Cosmic Physics
\def\gca{Geochim.~Cosmochim.~Acta}%
         % Geochimica Cosmochimica Acta
\def\grl{Geophys.~Res.~Lett.}%
         % Geophysics Research Letters
\def\jcp{J.~Chem.~Phys.}%
         % Journal of Chemical Physics
\def\jgr{J.~Geophys.~Res.}%
         % Journal of Geophysics Research
\def\jqsrt{J.~Quant.~Spec.~Radiat.~Transf.}%
         % Journal of Quantitiative Spectroscopy and Radiative Trasfer
\def\memsai{Mem.~Soc.~Astron.~Italiana}%
         % Mem. Societa Astronomica Italiana
\def\nphysa{Nucl.~Phys.~A}%
         % Nuclear Physics A
\def\physrep{Phys.~Rep.}%
         % Physics Reports
\def\physscr{Phys.~Scr}%
         % Physica Scripta
\def\planss{Planet.~Space~Sci.}%
         % Planetary Space Science
\def\procspie{Proc.~SPIE}%
         % Proceedings of the SPIE

%\newcommand\lcdm{\ifluatex \char"039B CDM\else\ifxetex\char"039B CDM \else%
%    \ensuremath{$\Lambda$\textrm{CDM}}\fi\xspace}

\def\head{ .ps \vbox to 0pt{\vss \hbox to 0pt{\hskip 440pt\rm
      LA-UR-10-07069\hss} \vskip 25pt}} 

\def \spose#1{\hbox  to 0pt{#1\hss}}  
\def \lta{\mathrel{\spose{\lower 3pt\hbox{$\sim$}}\raise 2.0pt\hbox{$<$}}}
\def \gta{\mathrel{\spose{\lower 3pt\hbox{$\sim$}}\raise 2.0pt\hbox{$>$}}}

\def\lcdm{\ensuremath{\Lambda\textrm{CDM}}\xspace}
    
\def\omegam{\ensuremath{\Omega_\textrm{m}}\xspace}
\def\omegal{\ensuremath{\Omega_\Lambda}\xspace}
\def\omegab{\ensuremath{\Omega_\textrm{b}}\xspace}
\def\omegar{\ensuremath{\Omega_\textrm{r}}\xspace}


%%%%%%%%%%%%%%%%%%%%%%%%%%%%%%%%%%%%%%%%%%%%%%%%%%%
\title[Early massive accretion events in MW-mass galaxies]{The influence of Gaia-Sausage-Enceladus-like merger events on the chemo-dynamics of galaxy discs}
%The influence of massive early merger events on the chemo-dynamics of galaxy discs

\author[T. Buck] {Tobias Buck$^{1,2}$\thanks{E-mail: tobias.buck@iwr-uni-heidelberg.de.de}, Aura Obreja$^{3}$, \etal \\  %Andrea V. Macci\`o$^{4,5}$, \etal \\
%, Aaron A. Dutton$^{2}$, Andrea V. Macci\`o$^{2,1}$\\
%
$^1$Universit\"at Heidelberg, Interdisziplin\"ares Zentrum f\"ur Wissenschaftliches Rechnen, Im Neuenheimer Feld 205, D-69120 Heidelberg, Germany\\
$^2$Universit\"at Heidelberg, Zentrum f\"ur Astronomie, Institut f\"ur Theoretische Astrophysik, Albert-Ueberle-Straße 2, 69120 Heidelberg, Germany\\
$^3$Universit\"ats-Sternwarte M\"unchen, Scheinerstraße 1, D-81679 M\"unchen, Germany%\\
%$^4$New York University Abu Dhabi, PO Box 129188, Saadiyat Island, Abu Dhabi, United Arab Emirates\\
%$^5$Max-Planck-Institut f\"ur Astronomie, K\"onigstuhl 17, 69117 Heidelberg, Germany\\
}

\setlength{\topmargin}{-1.2cm}

\begin{document}

\date{Accepted XXXX . Received XXXX; in original form XXXX}

\pagerange{\pageref{firstpage}--\pageref{lastpage}} \pubyear{2022}

\maketitle

\label{firstpage}


\begin{abstract}
awesome science! keep on reading and cite me! 
\end{abstract}

%%%%%%%%%%%%%%%%%%%%%%%%%%%%%%%%%%%%%%%%%%%%%%%%%%%
\noindent
\begin{keywords}

Galaxy: structure --- Galaxy: evolution --- Galaxy: kinematics and dynamics --- galaxies:
  formation --- Galaxy: disk --- methods: numerical
 \end{keywords}

%%%%%%%%%%%%%%%%%%%%%%%%%%%%%%%%%%%%%%%%%%%%%%%%%%%


%%%%%%%%%%%%%%%%%%%%%%%%%%%%%%%%%%%%%%%%%%%%%%%%%%%
\section{Introduction} \label{sec:introduction}
%%%%%%%%%%%%%%%%%%%%%%%%%%%%%%%%%%%%%%%%%%%%%%%%%%%

Our Milky Way (MW) Galaxy is the best-studied galaxy in the
universe and provides the most stringent constraints of galaxy formation
models \citep[e.g.,][]{Guedes11, Wetzel16, Grand17, Buck21}. Therefore, a thorough understanding of the components of the Galaxy, their dynamical evolution and formation channel is of great interest.

The Galaxy's main body, the rotationally-supported stellar disk harbors multiple components or populations when dissected in the space of chemistry, kinematics, spatial extent, and age \citep[e.g.,][]{Gilmore83, Norris85, Chiba00, Nissen10, Bovy12, Haywood13}; see \citet{Rix13} and \citet{Bland-Hawthorn16} for recent reviews. %There exist a dichotomy in both spatial structure and stellar chemistry. Spatially, MW's disk can be separated into a thin and a thick disk with different scale heights. Chemically, a clear bimodality in the abundance of $\alpha$ elements relative to iron ([$\alpha$/Fe]) has been found. In general, the high$-\alpha$ population has older ages and a larger scale-height than the low$-\alpha$ population. However, the mapping between the high$-\alpha$ (low$-\alpha$) and the thick (thin) disks is not perfect.

Thanks to the great improvements in the quality and volume of astrometric datasets provided by the {\it Gaia} mission in combination with chemical abundances and radial velocities from large spectroscopic surveys, (e.g. \textsc{apogee}; \textsc{galah}; \textsc{h3}; \textsc{LAMOST}; \citealt{Majewski2017}, \citealt{Martell2017}, \citealt{Conroy2019}, \citealt{Zhao_2012}, respectively) we are now able to identify and link coherent chemodynamical structures in the solar neighbourhood to ancient merger events several billion years back in MW's history (e.g. \citealt{Helmi2020} for a review). By now a large number of chemocynamical clusters associated with accreted systems have been identified in the high-dimensional space of orbital parameters and stellar abundances \citep[e.g.,][]{Belokurov18, Helmi18, Myeong19a, Naidu20a, Horta21, Myeong2022}.

Most notably, there exists an excess of stars on radial orbits in the local stellar halo around the Sun which is often referred to as {\it Gaia}-Sausage-Enceladus (GSE; \citealt{Belokurov2018, Helmi2018}; see also \citealt{Nissen2010, Koppelman2018, Haywood2018}). This stellar overdensity has been identified as the observable signature of a massive ($M_{\star} \simeq 10^{9}$ $\rm{M_{\odot}}$) early ($8-11$ Gyr ago; e.g. \citealt{Vin2019, Bel_2020, Naidu21, Xiang22}) accretion event. GSE is by far the most significant merger in MW's history and contributed approximately two-thirds of MW's stellar halo stars on highly-eccentric orbits \citep[e.g.][]{Mackereth+Bovy20}. GSE stars inhabit a sausage-like distribution in the radial-azimuthal velocity distribution \citep{Brook2003, Belokurov2018} and appear to be more metal-poor and less $\alpha$-enhanced than the redder halo counterpart \citep{Haywood_2018a, Helmi_2018}. The redder sequence of the Galactic halo is thought to be the result of proto-galactic disc stars being dynamically ejected into the halo during the GSE merger and has been associated with an event termed {\it the Splash} by \cite{Belokurov2020, Bonaca2020}, initially also found by other earlier studies \citep{Bonaca2017, Haywood2018, DiMatteo2019, Gallart2019}. 

The existence of the GSE merger and its strong impact on the structure of the proto-MW demands a thorough study of the processes that shape MW's chemo-dynamical evolution during such a merger in order to interpret the observational data at hand. In order to connect present day observables to Gyr old events we need tools like cosmological simulations that model galactic mass growth, mergers and subsequent star formation plus chemical enrichment self-consistently. Here, enough resolution to follow the internal disk dynamics and suppress spurious heating ($N \gtrsim 10^6$; e.g. \citealt{Sellwood2013,Ludlow2019DiskHeating, Ludlow2021}) as well as resolving the multi-phase, dense structure of the interstellar medium (ISM) combined with a proper model for chemical enrichment \citep[e.g.][]{Buck2021} is key to accurately capture their birth kinematics and the subsequent dynamical evolution of the disk (e.g. \citealt{House2011, Bird2013}).

Thanks to recent advances in computing power and progress in numerical methods modern cosmological zoom simulations now meet the aforementioned requirements and are able to model individual MW-mass galaxies by sampling varying environments and formation scenarios (e.g. \citealt{Sawala2016, Grand2017, Buck2020, Font2020, Applebaum2021, Agertz2020Vintergatan, Bird2021, Khoperskov2022-1InSitu, Wetzel2022}). This enables us to reconstruct our Galaxy's formation history by linking the occurrence of chemodynamical patterns at $z=0$ to specific events in the Galaxy's (e.g. \citealt{Bignone2019, Mackereth2019, Fattahi2019, Grand2020, Elias2020, Dillamore2022, Khoperskov2022-2Accreted, Khoperskov2022-3Metallicity, Pagnini2022, Rey2022}).

While GSE's strong impact on MW's stellar halo is by now well established, more recently studies have further investigated its defining influence on the formation of MW's thin and thick disk \citep[e.g.]{Grand2020,Ciuca2022,Orkney2022,Rey2023}. These studies suggest that the GSE merger strongly shaped the formation of MW's thick disk agreeing with thick disk formation scenarios from early simulation results by \cite{Brook_2004, Brook_2006}. The gas-rich merger has a two-fold effect on MW's stellar disk: (i) Firstly, it heats part of the existing proto-disc stars dynamically ejecting them into the halo and creating the Splash. (ii) Secondly, the merger provides fresh gas to the central galactic regions triggering a starburst that eventually forms the younger thick disc. After the merge, the thin disc then forms from the accretion of metal-poor gas in an inside-out, upside-down fashion \citep[e.g.,][]{Bird2013, Grand2018, Buck2020a}. \citet{Ciuca_2021} qualitatively confirmed this picture using APOGEE DR14 data and selecting stellar populations around the solar radius. 

Interestingly, when inferring the birth radii of MW disk stars, \citet{Lu2021} found a steepening of the metallicity gradient at the time of the GSE merger using data from LAMOST combined with {\it Gaia} eDR3. Here we use the NIHAO-UHD suite \citep{Buck2020} of cosmological hydrodynamical simulations of MW-mass galaxies in order to investigate the influence of early merger events on the chemo-dynamics of the stellar disk. By using 4 different simulations from the NIHAO-UHD suite we investigate independently sampled environments that follow 4 different formation scenarios. Especially we focus on the frequency of massive early mergers and their detailed impact on the formation of the stellar disk. This paper is structures as follows: In \S 2 we describe the simulations on which we base our analysis, followed by a presentation of our results in \S 3. In \S 4 we discuss and conclude our findings.

\begin{figure*}
    \script{merger_ratios.py}
    \begin{centering}
        \includegraphics[width=\linewidth]{figures/merger_ratios.pdf}
        \vspace*{-1.75em}
        \caption{Gas mass (blue line, left axis) and gas merger ratio (orange line, right axis) as a function of time. We identify several gas rich mergers that contribute around 10\% of gas to the main galaxy at various times of 2-4 Gyr and a late time merger starting around 9.5 Gyr that contributes 10\% of gas and continues for several peri-center passages and goes on for $\sim2$ Gyr.
        }
        \label{fig:merger_ratio}
    \end{centering}
\end{figure*}


%%%%%%%%%%%%%%%%%%%%%%%%%%%%%%%%%%%%%%%%%%%%%%%%%%%
\section{Methods} \label{sec:simulation}
%%%%%%%%%%%%%%%%%%%%%%%%%%%%%%%%%%%%%%%%%%%%%%%%%%%


For this work we use four simulations from the NIHAO-UHD suite \citep{Buck2020a} part of the Numerical Investigation of a Hundred Astronomical Objects (NIHAO) simulation suite \citep{Wang2015}. Parts of this simulation suite have previously been used to study the build-up of MW's peanut-shaped bulge \citep{Buck2018,Buck2019}, investigate the stellar bar properties \citep{Hilmi2020}, infer the MW's dark halo spin \citep{Obreja2022}, study the dwarf galaxy inventory of MW mass galaxies \citep{Buck2019a} or investigate the age-metallicity relation of MW disk stars \citep{Lu2022} including the chemical bimodality of disk stars \citep{Buck2020}, their abundances \citep{Lu2022a} and the origin of very metal-poor stars inside the stellar disk \citep{Sestito2021}.
Comparing the properties of these galaxies to observations of the MW and local disk galaxies from the SPARC data \citep{Lelli2016} \citet{Buck2020a} showed that simulated galaxy properties agree well with observations.

The simulations assume cosmological parameters from the \cite{Planck}, namely: \OmegaM= 0.3175, \OmegaL= 0.6825, \Omegab= 0.049, H${_0}$ = 67.1\kms\Mpc$^{-1}$, \sig8 = 0.8344. Initial conditions are created the same way as for the original NIHAO runs \citep[see][]{Wang2015} using a modified version of the \texttt{GRAFIC2} package \citep{Bertschinger2001,Penzo2014}. The mass resolution of these simulation ranges between $m_{\rm dark}\sim1.5 - 5.1\times10^5 \Msun$ for dark matter particles and $m_{\rm gas}\sim2.8 - 9.4\times10^4 \Msun$ for the gas particles. The corresponding force softenings are $\epsilon_{\rm dark}=414 - 620$ pc for the dark matter and $\epsilon_{\rm gas}=177 - 265$ pc for the gas and star particles. However, the adaptive smoothing length scheme implies that $h_{\rm  smooth}$ can be as small as $\sim30$ pc in the disk mid-plane. Stellar particles are born with an initial mass of $1/3\times m_{\rm{gas}}$ and are subject to massloss according to stellar evolution models as detailed in \citet{Stinson2013}. 
The simulation setup, star formation and feedback implementations are described in detail in the introductory paper \citep{Buck2020a} but for completeness we summarise them below. 

Simulations are performed with the modern smoothed particle hydrodynamics (SPH) solver {\texttt{GASOLINE2}} \citep{Wadsley2017} including substantial updates to the hydrodynamics as described in \citet{Keller2014}. {\texttt{GASOLINE2}} implements cooling via hydrogen, helium, and various metal-lines following \citet{Shen2010} using look-up tables calculated with \texttt{cloudy} \citep[version 07.02;][]{Ferland1998} and including photo-heating from the \citet{Haardt2005} UV background\footnote{For details on the impact of the UV background on galaxy formation see the recent study by \citet{Obreja2019}}. Star formation proceeds in cold (T $< 15,000$K), dense ($n_{\rm  th}  >  10.3$cm$^{-3}$) gas and is implemented as described in \citet{Stinson2006}. \citet{Buck2019a} showed that with this kind of star formation model only a high value of $n_{\rm  th}>10$cm$^{-3}$ \citep[see also][for an extended parameter study]{Dutton2019,Dutton2020} is able to reproduce the clustering of young star clusters as observed in the Legacy Extragalactic UV Survey (LEGUS) \citep{Calzetti2015,Grasha2017}.

Following \citet{Stinson2013} two modes of stellar feedback are implemented: (i) the energy input from young massive stars, e.g. stellar winds and photo ionization, prior to any supernovae explosions, thus termed \textit{early stellar feedback (ESF)}. This mode consists of the total stellar luminosity ($2 \times 10^{50}$ erg of thermal energy per $M_{\odot}$) of the entire stellar population with an efficiency for coupling the energy input of $\epsilon_{\rm ESF}=13\%$ \citep{Wang2015}. (ii) supernova explosions implemented using the blastwave formalism as described in \citet{Stinson2006} and making use of a delayed cooling formalism for particles inside the blast region following \citet{McKee1977} in order to account for the adiabatic expansion of the supernova.
Finally, we adopted a metal diffusion algorithm between particles as described in \citet{Wadsley2008}.


%\begin{figure}
%    \script{half_mass_radius_gas.py}
%    \begin{centering}
%        \includegraphics[width=\linewidth]{figures/279e12_gas_size.pdf}
%        \caption{
%            Half mass radius of the cold gas as a function of time. The sharp increases in the cold gas size around 2 Gyr and 9 Gyr indicate the influence of massive gas rich merger events on the galaxy. The addition of fresh cold gas leads to a sudden increase in cold gas half mass size.
%        }
%        \label{fig:half_mass}
%    \end{centering}
%\end{figure}

%%%%%%%%%%%%%%%%%%%%%%%%%%%%%%%%%%%%%%%%%%%%%%%%%%%
\section{Results} \label{sec:results}
%%%%%%%%%%%%%%%%%%%%%%%%%%%%%%%%%%%%%%%%%%%%%%%%%%%

\subsection{The impact of gas rich merger on the metallicity gradient} \label{sec:merger}
We start our analysis on the influence of Gaia-Sausage-Enceladus-like merger
events on the chemo-dynamics of MW-like galaxy disks by looking at the merger history of the four galaxies. Figure~\ref{fig:merger_ratio} shows the cumulative gas mass growth (blue line, left axis) accompanied by the gaseous merger ratio (orange line, right axis) defined as the ratio of gas mass of the merging satellite over the gas mass of the main galaxy. We see that after an initial phase ($0-\sim2.5$ Gyr) of rapid gas mass growth accompanied by violent, relatively gas rich mergers there are several other mergers at later times that contribute more than 10\% in gas leading to sudden jumps in the gas mass of the main galaxy (blue line). 
Especially for g2.79e12 (left most panel) and g7.08e11 (right most panel) we see some late time ($\sim10$ Gyr) gas rich merger while the other two galaxies do not show any gas rich merger after $5$ Gyr for g8.26e11 and $7$ Gyr for g7.55e11.

\begin{figure*}
    \script{half_mass_radius_metal_gradient.py}
    \begin{centering}
     \includegraphics[width=\linewidth]{figures/half_mass_radius_metal_gradient.pdf}
        \vspace*{-1.75em}
        \caption{
            Half mass radius (orange line, right axis) and metallicity gradient (blue line, left axis) of the cold gas as a function of time. The sharp increases in the cold gas size around the times of gas rich mergers coincide with a sudden steepening of the metallicity gradient of $\sim0.02$ dex/kpc. The addition of relatively unenriched cold gas by the merging satellite leads to a sudden increase of the star-forming cold gas  mass in the outskirts of the galaxies while the central parts are unaffected. This steepens the gradient.
        }
        \label{fig:half_mass}
    \end{centering}
\end{figure*}

The sudden increase of gas mass due to these gas rich merger leads to a sudden increase in the half mass radius ($\gsim20$ kpc) of the cold gas disk of the main galaxy as the orange line (right axis) in Figure~\ref{fig:half_mass} shows. Simultaneously we plot the metallicity gradient of the cold gas disk of the main galaxy as a function of time (blue line in Fig.~\ref{fig:half_mass}, left axis) as measured by fitting a straight line to the radial metallicity profile omitting the central $2.5$ kpc. In order to minimize short time scale fluctuations we plot the running average with a window size of 10 corresponding to a time window of $\sim600$Myr. 

We find that the time evolution of the metallicity gradient in the galaxies shows a rather complex behaviour. Galaxies g2.79e12 and g8.26e11 show a fast flattening of the initial metallicity gradient at early times ($\lesssim2$ Gyr and $\lesssim 2.5$ Gyr) from initially $-0.08$ dex/kpc to $-0.04$ dex/kpc while g7.55e11 and g7.08e11 start out already with a shallow gradient of $\sim-0.05$ and $\sim-0.04$.
Common to all four galaxies we find a steepening of the metallicity gradient by at least $0.02$ dex/kpc over a time frame of $\sim1$ Gyr at times where the gas disk size rapidly increases (marked with gray vertical lines) due to the gas mass increase of the merging satellites. After the steepening happened, the gradient flattens again by $\sim0.01-0.02$ dex/kpx over roughy the same time frame except for g2.79e12 for which the gradient continues to steepen after the merger up to a maximum of $\sim-0.1$ dex/kpc. Only after another $\sim1.5$ Gyr the gradient starts to quickly flatten again until the present day. The late time evolution of the gradient on all four galaxies is comparatively weak but it consistently continues to flatten with at most a decrease of the gradient of $\sim0.02$ dex/kpc over a time of $\sim6-7$ Gyr.


\begin{figure}
    \script{gas_profile_all.py}
    \begin{centering}
        \includegraphics[width=\linewidth]{figures/gas_profile_2d.pdf}
        \caption{
            Relative change in cold gas surface density profile before and after the merger events as indicated in Fig.~\ref{fig:merger_ratio} by the gray lines. Outside of a radius of $\sim5-10$ kpc the merging satellites contribute significant amounts of cold gas up to 200\%.
        }
        \label{fig:surf_den}
    \end{centering}
\end{figure}

\subsection{The physical mechanism behind a steepening of the metallicity gradient}
\label{sec:steepening}

A sudden steepening of the galactic metallicity gradient is unexpected and has been deemed not possible (reference??? Ivan, any clues?). The physical reason for a steepening of the metallicity gradient might be a dilution of the outer disk metallicity by the fresh, un-enriched gas brought in by the merging satellite, the continued or preferential enrichment of the galactic center or a combination of both. A dilution scenario might be expected from the general lower metallicity of low mass satellites compared to MW-mass galaxies owing to the shape of the stellar mass-metallicity relation and has previously been suggested for the creation of the chemical bimodality of the MW's stellar disk \citep[e.g.][]{Buck2020}.

In order to investigate the cause of the strong steepening of the metallicity gradient in our simulations we first plot in Figure~\ref{fig:surf_den} where in the disk the accreted gas mostly contributes. Figure~\ref{fig:surf_den} plots the relative change of cold gas surface mass density radial profile before the gas rich merger and after the merger finished as indicated by the gray lines in Figure~\ref{fig:half_mass}. This figure reveals an interesting finding. Consistently across all four galaxies the surface mass density of cold gas in the inner $5$ kpc is roughly constant or decreases slightly due to gas consumption by star formation and gas expulsion by feedback. Outside of $5$ kpc the surface mass density increases by $\sim75$ \% up to $\sim200$\% due to cold gas added byt he merger event, either by direct accretion of cold gas from the merging satellite or by triggered gas accretion from the circum-galactic medium due to tidal forces exerted by the interacting satellite.

Three scenarios can now lead to a steepenig of the metallicity gradient: (i) the metallicity of the star forming gas in the outskirts of the galaxies is diluted by the freshly accreted gas; (ii) the central parts of the galaxy keep enriching more strongly than the outskirts or (iii) a mix of both.
In fact, as was shown in \citet{Buck2020} in their Fig. 8 scenario (iii) is happening in these galaxies. For most of the time the ISM/cold gas metallicity evolution is self-similar and almost monotonically increasing at each radius but with markedly dilution events that correlate with merger events. In order to investigate this in more depth and study its effect on the metallicity gradient, we plot in Fig.~\ref{fig:feh_evolution} the difference in metallicity of the cold gas, $\Delta\mathrm{[Fe/H]}=\mathrm{[Fe/H]}(R<2\, \mathrm{kpc})-\mathrm{[Fe/H]}(R)$, between the gas in a given annulus at a radius $R$ and the metallicity in the central parts ($R<2$ kpc) of the galaxy, $\mathrm{[Fe/H]}(R<2\, \mathrm{kpc})$ as a function of time. We normalize this difference in metallicity, $\Delta\mathrm{[Fe/H]}$, by the central metallicity in order to take out the continued enrichment in [Fe/H] and highlight deviations from the self-similar evolution which is indicating a steepening or flattening of the metallicity gradient as measured over different radial ranges. Thus, a flat line in Fig.~\ref{fig:feh_evolution} would indicate perfect self-similar evolution while a more negative value indicates that a certain radius lacks behind the central galaxy in [Fe/H] enrichment and thus a steepening of the metallicity gradient. On the other hand, a progressively more positive or less negative value shows that the radius at hand enriches more strongly in metallicity than the central parts an thus indicates a flattening of the metallicity gradient.




\textcolor{red}{make same figure for relative metallicity change or similar to buck2020 Fig. 8 metallicity in different radii but normalized to central bin}

%\begin{figure}
%    \script{feh_gradient.py}
%    \begin{centering}
%        \includegraphics[width=\linewidth]{figures/feh_gradient.pdf}
%        \caption{
%            Metallicity gradient of the cold gas disk during the merger events. We find that the contribution of additional metal poor gas lowers the metallicity in the outskirts of the disk effectively steepening the metallicity gradient by $\sim 0.1$ dex.
%        }
%        \label{fig:feh_grad}
%    \end{centering}
%\end{figure}

\begin{figure*}
    \script{enrichment_evolution.py}
    \begin{centering}
        \includegraphics[width=\linewidth]{figures/enrichment_evolution.pdf}
        \caption{
            some caption
        }
        \label{fig:feh_evolution}
    \end{centering}
\end{figure*}

%\begin{figure}
%    \script{v_phi.py}
%    \begin{centering}
%        \includegraphics[width=\linewidth]{figures/2.79e12_v_phi_gas.pdf}
%        \caption{
%            Histogram of the cold gas velocity in the disk plane, $v_\phi$, before and after the merger. The peak of the $v_\phi$ distribution shifts from 73 km/s before the merger to 140 km/s after the merger, almost doubling the rotation speed of the cold gas disk.
%        }
%        \label{fig:v_phi}
%    \end{centering}
%\end{figure}

\subsection{The impact on the age-metallicity relation and the [O/Fe] vs. [Fe/H] plane}
\label{sec:g2.79e12}

We now turn to investigate the question where the early gas rich mergers appear in the age metallicity relation and the [O/Fe] vs. [Fe/H] plane. To this extent we focus on the galaxy g2.79e12 for which the steepening of the metallicity gradient is strongest. Figure~\ref{fig:mdf} shows the metallicity distribution function (MDF) for gas (orange histogram) and stars (blue histogram) of three gas rich satellites that merge with the main galaxy at $2.02$ Gyr and $2.57$ Gyr in comparison to the MDF of gas (gray histogram) and stars (black histogram) in the central galaxy. 

\begin{figure*}
    \script{mdf.py}
    \begin{centering}
        \includegraphics[width=\linewidth]{figures/2.79e12_mdf_gas.pdf}
        \caption{
            Metallicity distribution function of the main galaxy and the merging satellite shortly before coalescence for all three merger events. Filled histograms show the gaseous MDF while steps show the MDF for stars. In any case, the gas metallicity of the satellite is $\sim0.5-0.75$ dex lower that the main galaxy's gas metallicity.
        }
        \label{fig:mdf}
    \end{centering}
\end{figure*}





\begin{figure*}
    \script{age_feh.py}
    \begin{centering}
        \includegraphics[width=\linewidth]{figures/2.79e12_age_metallicity_grid.pdf}
        \caption{
            Age-metallicity relation.
        }
        \label{fig:age_feh}
    \end{centering}
\end{figure*}

\begin{figure*}
    \script{feh_ofe.py}
    \begin{centering}
        \includegraphics[width=\linewidth]{figures/2.79e12_feh_ofe_grid.pdf}
        \caption{
            Oxygen vs. metallicity.
        }
        \label{fig:ofe_feh}
    \end{centering}
\end{figure*}



%%%%%%%%%%%%%%%%%%%%%%%%%%%%%%%%%%%%%%%%%%%%%%%%%%%
\section{Conclusion}
\label{sec:conclusion}
%%%%%%%%%%%%%%%%%%%%%%%%%%%%%%%%%%%%%%%%%%%%%%%%%%%



Our results are summarized as follows:
\begin{itemize}
%
\item 

\end{itemize}



%%%%%%%%%%%%%%%%%%%%%%%%%%%%%%%%%%%%%%%%%%%%%%%%%%%
\section*{acknowledgments}
%\section*{Acknowledgments}
%%%%%%%%%%%%%%%%%%%%%%%%%%%%%%%%%%%%%%%%%%%%%%%%%%%
TB's contribution to this project was made possible by funding from the Carl Zeiss Foundation. TB gratefully acknowledges the Gauss Centre for Supercomputing e.V. (www.gauss-centre.eu) for funding this project by providing computing time on the GCS Supercomputer SuperMUC at Leibniz Supercomputing Centre (www.lrz.de). This research was carried out on the High Performance Computing resources at New York University Abu Dhabi.
This research made use of the {\sc{pynbody}} \citet{pynbody} package to analyze the simulations and used the {\sc{python}} package {\sc{matplotlib}} \citep{matplotlib} to display all figures in this work. Data analysis for this work made intensive use of the {\sc{python}} library {\sc{SciPy}} \citep{scipy}, in particular {\sc{NumPy and IPython}} \citep{numpy,ipython}. The article has been typeset using showyourwork! by \citet{Luger2021}.


\bibliography{bib.bib}

\appendix

%\section{Rotation velocity of stars}

%\begin{figure}
%    \script{v_phi_stars.py}
%    \begin{centering}
%        \includegraphics[width=\linewidth]{figures/2.79e12_v_phi_stars.pdf}
%        \caption{
%            Same as Fig.~\ref{fig:v_phi} but for the stellar rotation velocity in the disk plane, $v_\phi$, before and after the merger. The peak of the $v_\phi$ distribution shifts from 27 km/s before the merger to 66 km/s after the merger, more than doubling the rotation speed of the stellar disk.
%        }
%        \label{fig:v_phi_stars}
%    \end{centering}
%\end{figure}

\section{Oxygen abundance distribution}

\begin{figure*}
    \script{mdf_oxygen.py}
    \begin{centering}
        \includegraphics[width=\linewidth]{figures/2.79e12_mdf_oxygen_gas.pdf}
        \caption{
            Same as Fig.~\ref{fig:mdf} but for the oxygen abundance. Filled histograms show the gaseous oxygen abundance distribution while steps show the one for stars. The gas oxygen abundance of the satellite is  $\sim0.1$ dex lower that the main galaxy's gas oxygen abundance.
        }
        \label{fig:mdf_oxygen}
    \end{centering}
\end{figure*}



\label{lastpage}
\end{document}

